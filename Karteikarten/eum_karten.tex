% Karteikarten mit Formeln aus der Vorlesung Elektrizität und Magnetismus
% bei Prof. Schrag im SS24
% an der TU München
%
%		* ----------------------------------------------------------------------------
%		* "THE BEER-WARE LICENSE" (Revision 42/023):
%		* Ronny Bergmann <mail@darkmoonwolf.de> wrote this file. As long as you retain
%		* this notice you can do whatever you want with this stuff. If we meet some day,
%		* and you think  this stuff is worth it, you can buy me a beer or a coffee in return.
%		* ----------------------------------------------------------------------------
%

\documentclass[a7paper]{kartei}
\usepackage{amsmath}
\usepackage{amssymb}
\usepackage[utf8]{inputenc} %UTF8

\begin{document}

\setlength{\parindent}{0pt} %Am Anfang einer neuen Zeile nicht einrücken

% Karte 1: Coulomb’sches Gesetz
\begin{karte}[Elektrostatik]{Coulomb’sches Gesetz}
Kraft auf \( q \) durch \( q_i \)

\[
\vec{F}(\vec{r}) = \sum_{i=1}^{N} \frac{q q_i}{4\pi\epsilon} \cdot \frac{\vec{r} - \vec{r}_i}{|\vec{r} - \vec{r}_i|^3}
\]

\([F] = \text{N} = \frac{\text{kg} \cdot \text{m}}{\text{s}^2}\)
\end{karte}

% Karte 2: Elektrische Feldstärke
\begin{karte}[Elektrostatik]{Elektrische Feldstärke}
Feld, durch das \( q \) Kraft \( \vec{F} \) erfährt

\[
\vec{E}(\vec{r}) = \frac{\vec{F}(\vec{r})}{q}
\]

\([E] = \frac{\text{V}}{\text{m}} = \frac{\text{kg} \cdot \text{m}}{\text{A} \cdot \text{s}^3}\)
\end{karte}

% Karte 3: Elektrische Arbeit
\begin{karte}[Elektrostatik]{Elektrische Arbeit}
Arbeit, um Punktladung von \( P_1 \) nach \( P_2 \) zu bringen

\[
W_{12} = \int_{C(P_1, P_2)} \vec{F}(\vec{r}) \, \mathrm{d}\vec{r}
\]

\([W] = \text{J} = \text{Nm} = \frac{\text{kg} \cdot \text{m}^2}{\text{s}^2}\)
\end{karte}

% Karte 4: Grundgesetz der Elektrostatik
\begin{karte}[Elektrostatik]{Grundgesetz der Elektrostatik}
Elektrostatische Felder sind konservativ:

\[
\oint_C \vec{E} \, \mathrm{d}\vec{r} = 0
\]

\[
\int_{P_1}^{P2} \vec{E} \, \mathrm{d}\vec{r} = \text{const.}
\]

\[\text{rot} \, \vec{E} = \nabla \times \vec{E} = 0\]

\end{karte}

% Karte 5: Elektrisches Potential
\begin{karte}[Elektrostatik]{Elektrisches Potential}

\[
\Phi(\vec{r}) = \Phi(\vec{r_0}) - \int_{P_0}^{P} \vec{E}(\vec{r}\ ') \, \mathrm{d}\vec{r} \ '
\]
\\
\(
\vec{r}_0 \text{: Referenzpotential (meist 0)}
\)
\\

\([\Phi] = \text{V} = \frac{\text{kg} \cdot \text{m}^2}{\text{A} \cdot \text{s}^3}\)
\end{karte}

% Karte 6: Spannung zwischen zwei Punkten
\begin{karte}[Elektrostatik]{Spannung zwischen zwei Punkten}
\[
U_{12} = \int_{P_1}^{P_2} \vec{E} \, \mathrm{d}\vec{r}
\]

\[
U_{12} = \Phi(P_1) - \Phi(P_2) = -U_{21}
\]

\([U] = \text{V} = \frac{\text{kg} \cdot \text{m}^2}{\text{A} \cdot \text{s}^3}\)
\end{karte}

% Karte 7: Elektrische Feldkonstante
\begin{karte}[Elektrostatik]{Elektrische Feldkonstante}
\[
\epsilon = \epsilon_0 \epsilon_r
\]

\[
\epsilon_0 = 8.854188 \times 10^{-12} \, \frac{\text{As}}{\text{Vm}}
\]

\[
\epsilon_0 = 8.854188 \times 10^{-12} \, \frac{\text{A}^2\text{s}^4}{\text{kg} \cdot \text{m}^3}
\]
\end{karte}

% Karte 8: Elektrische Flussdichte
\begin{karte}[Elektrostatik]{Dielektrisches Verschiebungsfeld}
\[
\vec{D}(\vec{r}) = \epsilon_0 \epsilon_r \vec{E}(\vec{r})
\]

\([D] = \frac{\text{As}}{\text{m}^2}\)
\end{karte}

% Karte 9: Raumladungsdichte
\begin{karte}[Elektrostatik]{Raumladungsdichte}
\[
Q(V) = \int_V \rho(\vec{r}) \, \mathrm{d}^3r
\]

\([\rho] = \frac{\text{As}}{\text{m}^3}\)
\end{karte}

% Karte 10: Oberflächenladungsdichte
\begin{karte}[Elektrostatik]{Oberflächenladungsdichte}
\[
Q(S) = \int_S \sigma(\vec{r}) \, \mathrm{d}a
\]

\[
\sigma = \vec{D} \cdot \vec{N}
\]

\([\sigma] = \frac{\text{As}}{\text{m}^2}\)
\end{karte}

% Karte 11: Gaußsches Gesetz
\begin{karte}[Elektrostatik]{Gaußsches Gesetz}
\center 1. Maxwellgleichung
\\
\[
\int_{\partial V} \vec{D} \, \mathrm{d}\vec{a} = Q(V) = \int_{V} \rho \, \mathrm{d}V
\]

\[
\text{div} \, \vec{D} = \rho
\]

Herleitung mit Satz von Gauß.
\end{karte}

% Karte 12: Poissongleichung
\begin{karte}[Elektrostatik]{Poissongleichung}
\[
\text{div} (\epsilon \nabla \Phi) = -\rho
\]
\end{karte}

% Karte 13: Kapazität
\begin{karte}[Elektrostatik]{Kapazität}
\[
C = \frac{Q}{U_{12}}
\]

\([C] = \text{F} = \frac{\text{As}}{\text{V}} = \frac{\text{A}^2\text{s}^4}{\text{kg} \cdot \text{m}^2}\)
\end{karte}

% Karte 14: Kondensatoren
\begin{karte}[Elektrostatik]{Kondensatoren}
Plattenkondensator:
\[
C_P = \epsilon \frac{A}{d}
\]

Kugelkondensator:
\[
C_K = 4\pi \epsilon \frac{ab}{b - a}
\]
\end{karte}

% Karte 15: Kondensatorschaltungen
\begin{karte}[Elektrostatik]{Kondensatorschaltungen}
Parallelschaltung:
\[
C_{\text{par}} = \sum_{i=1}^{N} C_i
\]

Reihenschaltung:
\[
\frac{1}{C_{\text{rei}}} = \sum_{i=1}^{N} \frac{1}{C_i}
\]
\end{karte}

% Karte 16: Dielektrika
\begin{karte}[Elektrostatik]{Dielektrika}
Parallelschaltung:
\[
C_P = C_1 + C_2 = \frac{\epsilon_1 A_1}{d} + \frac{\epsilon_2 A_2}{d}
\]

Reihenschaltung:
\[
C_R = \left(\frac{1}{C_1} + \frac{1}{C_2}\right)^{-1} = \left(\frac{d_1}{\epsilon_1 A} + \frac{d_2}{\epsilon_2 A}\right)^{-1}
\]
\end{karte}

% Karte 17: Energie eines Kondensators
\begin{karte}[Elektrostatik]{Energie eines Kondensators}
\[
W_{el} = \frac{1}{2} Q U = \frac{1}{2} C U^2
\]

\[
W_{el} = \int_V w_{el} \, \mathrm{d}V
\]

\([W] = \text{J} = \frac{\text{kg} \cdot \text{m}^2}{\text{s}^2}\)
\end{karte}

% Karte 18: Energiedichte des E-Felds
\begin{karte}[Elektrostatik]{Energiedichte des E-Felds}
\[
w_{el} = \frac{1}{2} \vec{E} \cdot \vec{D}
\]

\([w] = \frac{\text{J}}{\text{m}^3} = \frac{\text{kg}}{\text{m} \cdot \text{s}^2}\)
\end{karte}

% Karte 19: Stromstärke
\begin{karte}[Stationäre Ströme]{Stromstärke}
\[
I = \frac{\mathrm{d}Q}{\mathrm{d}t}
\]

\([I] = \text{A}\)
\end{karte}

% Karte 20: Stromdichte
\begin{karte}[Stationäre Ströme]{Stromdichte}
\[
I(S) = \int_S \vec{j} \, \mathrm{d}\vec{a}
\]

\[
\vec{j} = q \cdot n \cdot \vec{v} = \rho \cdot \vec{v} = |q| \cdot n \cdot \mu \cdot \vec{E}
\]

\([j] = \frac{\text{A}}{\text{m}^2}\)
\end{karte}

% Karte 21: Ohmsches Gesetz
\begin{karte}[Stationäre Ströme]{Ohmsches Gesetz}
\[
\vec{j} = \sigma \vec{E}
\]

\[
I = G U
\]
\end{karte}

% Karte 22: Verlustleistung
\begin{karte}[Stationäre Ströme]{Verlustleistung}
\[
P_{el} = U I = \frac{U^2}{R} = R I^2
\]

\([P] = \text{W} = \text{V} \cdot \text{A} = \frac{\text{kg} \cdot \text{m}^2}{\text{s}^3}\)
\end{karte}

% Karte 23: Lorentzkraft
\begin{karte}[Magnetostatik]{Lorentzkraft}
\[
\vec{F}_{L} = q \cdot (\vec{v} \times \vec{B})
\]
\[
\mathrm{d}\vec{F}_{L} = I \cdot \mathrm{d}\vec{s} \times \vec{B}
\]
\\
Elektromagnetische Kraft:
\[
\vec{F}_{em} = q (\vec{E} + \vec{v} \times \vec{B})
\]
\\
\end{karte}

% Karte 24: Drehmoment
\begin{karte}[Magnetostatik]{Drehmoment}
\[
\vec{m} = I\vec{A}
\]

\[
\vec{M} = \vec{m} \times \vec{B} = I \vec{A} \times \vec{B}
\]

\([M] = \text{Nm} = \frac{\text{kg} \cdot \text{m}^2}{\text{s}^2}\)
\end{karte}

% Karte 25: Quellenfreiheit des B-Feldes
\begin{karte}[Magnetostatik]{Quellenfreiheit des B-Feldes}
\center 3. Maxwellgleichung
\\
\[
\int_{\partial V} \vec{B} \, \mathrm{d}\vec{a} = 0\]

\[
\text{div} \, \vec{B} = 0
\]

\([B] = \text{T} = \frac{\text{V} \cdot \text{s}}{\text{m}^2} = \frac{\text{kg}}{\text{A} \cdot \text{s}^2}\)
\end{karte}

% Karte 26: Ampèresches Durchflutungsgesetz
\begin{karte}[Magnetostatik]{Ampèresches Durchflutungsgesetz}
\center 4. Maxwellgleichung
\\
\[
\int_{\partial A} \vec{H} \, \mathrm{d}\vec{s} = \int_A \left(\vec{j} + \frac{\partial \vec{D}}{\partial t}\right) \, \mathrm{d}\vec{a}
\]

\[
\text{rot} \, \vec{H} = \vec{j} + \frac{\partial \vec{D}}{\partial t}
\]

\end{karte}

% Karte 27: Magnetische Feldkonstante
\begin{karte}[Magnetostatik]{Magnetische Feldkonstante}
\[
\mu = \mu_0 \mu_r
\]

\[
\mu_0 = 4\pi \times 10^{-7} \, \frac{\text{Vs}}{\text{Am}}
\]

\[
\mu_0 = 12.56637 \times 10^{-7} \, \frac{\text{kg} \cdot \text{m}}{\text{A}^2 \text{s}^2}
\]
\end{karte}

% Karte 28: Magnetische Feldstärke
\begin{karte}[Magnetostatik]{Magnetische Feldstärke}
\[
\vec{H} = \frac{1}{\mu} \vec{B}
\]

\([H] = \frac{\text{A}}{\text{m}}\)
\end{karte}

% Karte 29: Magnetischer Fluss
\begin{karte}[Magnetostatik]{Magnetischer Fluss}
\[
\Phi = \int_A \vec{B} \, \mathrm{d}\vec{a}
\]

\([\Phi] = \text{Vs} = \frac{\text{kg} \cdot \text{m}^2}{\text{A} \cdot \text{s}^2}\)
\end{karte}

% Karte 30: Induktionsgesetz
\begin{karte}[Magnetostatik]{Induktionsgesetz}
\center 2. Maxwellgleichung
\\
\[
\text{rot} \, \vec{E} = -\frac{\partial \vec{B}}{\partial t}
\]
\end{karte}

% Karte 31: Induzierte Spannung
\begin{karte}[Magnetostatik]{Induzierte Spannung}
\textbf{Ruheinduktion}:

\[
U_{ind} = -\int_{A(t)} \frac{\partial \vec{B}}{\partial t} \, \mathrm{d}\vec{a} + \int_{\partial A} (\vec{v} \times \vec{B}) \, \mathrm{d}\vec{r}
\]
\textbf{Bewegungsinduktion}:
\[
U_{ind} = -\frac{\mathrm{d}}{\mathrm{d}t} \Phi(A(t))
\]
\end{karte}

\end{document}
